\documentclass[letterpaper,11pt]{article}
\usepackage[brazil]{babel}
\usepackage[utf8]{inputenc}
\usepackage{amsmath}
\usepackage{amssymb}
\usepackage{graphicx}
\newtheorem{theorem}{Teorema}[section]

\title{MO420 - TP de Geração de Colunas}
\author{RA009206 - Luís Guilherme Fernandes Pereira \\
RA044072 - Igor Ribeiro de Assis}
\date{Trabalho 2 - 1o semestre de 2009}

\begin{document}
\maketitle

\section{Introdução}

Neste trabalho prático, estudamos o problema de geração de colunas,
implementamos essa técnica no resolvedor XPress e usamo-la para resolver
o problema de \emph{multi-item capacitated lot sizing} (MILSP).

\section{Descrição}
Este trabalho consistiu nas seguintes atividades:

\begin{enumerate}
 \item Implementação de um algoritmo de geração de colunas para
       resolução do MILSP. Para isso foram necessárias uma política de 
       geração de colunas iniciais e uma regra para geração de colunas
       posteriores.
 \item Geração de limitantes duais a partir da resolução dos problemas
       mestre restritos.
 \item Formulação inteira do ULS, que foi utilizado como problema de
       pricing.
 \item Resolução determinística do ULS, utilizando o algoritmo de
       programação dinâmica de Wagner e Within. 
\end{enumerate}

\section{Implementação}
A política de geração de colunas iniciais foi...

Geramos limitantes duais...

A formulação inteira para o ULS foi da forma:
minimize
\begin{equation}
 \sum_{t=1}^n p_tx_t + \sum_{t=1}^n h_ts_t + \sum_{t=1}^n f_ty_t
\end{equation}
s.a
\begin{equation}
 s_{t-1} + x_t = d_t + s_t \forall t \in \{1..n\}
\end{equation}
\begin{equation}
 x_{t} \leq My_t \forall t \in \{1..n\}
\end{equation}
\begin{equation}
 s_0 = 0, s_t, x_t \geq 0, y_t \in {0,1}
\end{equation}

O algoritmo de Wagner e Within foi implementado em $O(n^3)$, utilizando
ideias presentes no algoritmo de Floyd-Warshall para grafos.  

Preocupamo-nos em escrever um código que fosse reutilizável, isto é,
dado um novo par ``problema mestre''-``problema restrito'', basta
implementar duas classes em C++ segundo certa interface padrão, e
poderemos utilizar nosso código sem outras alterações para fazer a
geração de colunas via XPress.

\section{Análise de resultados}

\section{Nota final}


\bibliographystyle{acm}
\bibliography{relatorio}


\end{document}

