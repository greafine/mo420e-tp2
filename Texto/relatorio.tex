\documentclass[letterpaper,11pt]{article}
\usepackage[brazil]{babel}
\usepackage[utf8]{inputenc}
\usepackage{amsmath}
\usepackage{amssymb}
\usepackage{graphicx}
\newtheorem{theorem}{Teorema}[section]

\title{MO420 - TP de Geração de Colunas}
\author{RA009206 - Luís Guilherme Fernandes Pereira \\
RA044072 - Igor Ribeiro de Assis}
\date{Trabalho 2 - 1o semestre de 2009}

\begin{document}
\maketitle

\section{Introdução}

Neste trabalho prático, estudamos o problema de geração de colunas,
implementamos essa técnica no resolvedor XPress e usamo-la para resolver
o problema de \emph{multi-item capacitated lot sizing} (MILSP).

\section{Descrição}
Este trabalho consistiu nas seguintes atividades:

\begin{enumerate}
 \item Implementação de um algoritmo de geração de colunas para
       resolução do MILSP. Para isso foram necessárias uma política de 
       geração de colunas iniciais e uma regra para geração de colunas
       posteriores.
 \item Geração de limitantes duais a partir da resolução dos problemas
       mestre restritos.
 \item Formulação inteira do ULS, que foi utilizado como problema de
       pricing.
 \item Resolução determinística do ULS, utilizando o algoritmo de
       programação dinâmica de Wagner e Within. 
\end{enumerate}

\section{Implementação}
A política de geração de colunas iniciais foi da seguinte forma:
geramos uma matriz identidade para termos uma base viável, e outros
$M*T$ esquemas, um para cada produto em que no esquema $t$ do produto
$k$ é produzida toda a demanda dos períodos de $t..T$ e nos períodos
de $1..t-1$ é produzida a demanda do período. Observe que da forma
como as colunas iniciais são geradas não necessiariamente se tem uma
solução do MILSP\footnote{As colunas (por produto) formam uma matriz
  triangular superior, portanto a chance de conflitos grande.}.

Note que (na maioria dos casos) as colunas da matriz identidade não
formam esquemas de produção e portanto não são válidas, por isso a
estas colunas são atribuídos custos bastante altos de forma que sejam
escolhidas para sair da base conforme colunas vão sendo geradas pelo
\emph{pricing}. Vale dizer também que mesmo as colunas geradas pelo
pricing podem não ser válidas já que estas são soluções do ULS e podem
violar a capacidade de produção $C_t$ da máquina.

Geramos limitantes duais quando o valor da solução do problema de
pricing é igual a zero, isto é, não existe coluna que possa entrar na
base e portanto chegamos ao ótimo do problema mestre relaxado. Também,
a cada solução inteira, temos limitantes primais.

A formulação inteira para o ULS foi da forma:

\begin{align}
  min \hspace{0.2cm}&  \sum_{t=1}^n p_tx_t + \sum_{t=1}^n h_ts_t + \sum_{t=1}^n
  f_ty_t \notag\\
  s.a \hspace{0.2cm}&  s_{t-1} + x_t = d_t + s_t &\forall t \in \{1..n\} \notag\\
      &  x_{t} \leq My_t &\forall t \in \{1..n\} \notag\\
      &  s_0 = 0, s_t, x_t \geq 0, y_t \in \{0,1\} \notag
\end{align}

A formulação utilizada não é a melhor no sentido de que há formulações
compactas do ULS que descrevem a envoltória convexa deste problema
\cite{wolsey2000ip}, no entanto a escolha desta formulação se deve ao
fato de ser mais direta a transformação de sua solução em planos de
produção (colunas) da formulação do MILSP utilizada.

Para resolver o ULS foi utilizada o resolvedor de MIP do XPress, aqui
é um ponto que seria vantajoso utilizar a formulação que descreve a
envoltória convexa pois poderia ser utilizado um resolvedor de
programação linear (Simplex) mais rápido que um de MIP já que na
maioria dos casos os resolvedors de PLI utilizam algoritmos como o
Simplex em subrotinas que calculam limitantes.

%% \begin{equation}
%%   min
%%  \sum_{t=1}^n p_tx_t + \sum_{t=1}^n h_ts_t + \sum_{t=1}^n f_ty_t
%% \end{equation}

%% \begin{equation}
%%   s.a s_{t-1} + x_t = d_t + s_t \forall t \in \{1..n\}
%% \end{equation}
%% \begin{equation}
%%  x_{t} \leq My_t \forall t \in \{1..n\}
%% \end{equation}
%% \begin{equation}
%%  s_0 = 0, s_t, x_t \geq 0, y_t \in {0,1}
%% \end{equation}

Como se conhece uma formulação compacta que descreve a envoltória
convexa do ULS é de se esperar que se tenham algoritmos combinatórios
para resolvê-lo, que é o caso. O algoritmo de Wagner e Within foi
implementado em $O(n^2)$.

Preocupamo-nos em escrever um código que fosse reutilizável, isto é,
dado um novo par ``problema mestre''-``problema restrito'', basta
implementar duas classes em C++ segundo certa interface padrão, e
poderemos utilizar nosso código sem outras alterações para fazer a
geração de colunas via XPress.

\section{Análise de resultados}
Muito provavelmente devido à escolha inicial de colunas, muitos testes
apresentaram resultados muito altos. No diretório Testes do trabalho,
além da saída completa, poder-se-á ver a evolução das iterações e o
resultado final. 

Comparando resultados de solução optima entre T1 e T2 encontramos
resultados muitas vezes idênticos. Contudo, T1 não nos dá a garantia de
uma solução ótima. Infelizmente não conseguimos fazer testes mais
aprofundados com T2 para comparar os resultados.

\bibliographystyle{alpha}
\bibliography{relatorio}

\end{document}
